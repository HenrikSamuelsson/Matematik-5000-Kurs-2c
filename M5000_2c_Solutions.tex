\documentclass {article}
\usepackage {geometry}
\usepackage {fancyhdr}
\usepackage {amsmath, amsthm, amssymb}
\usepackage {graphicx}
\usepackage {hyperref }
\usepackage {enumerate}
\usepackage[T1]{fontenc}
\usepackage[utf8]{inputenc}

\newenvironment{absolutelynopagebreak}
  {\par\nobreak\vfil\penalty0\vfilneg
   \vtop\bgroup}
  {\par\xdef\tpd{\the\prevdepth}\egroup
   \prevdepth=\tpd}


\begin{document}

\begin{absolutelynopagebreak}
\noindent {\bf 1103 }
\begin{enumerate}[a)]
\item $(5x + 2y) + (2x + y) = 5x + 2y + 2x + y = 7x + 3y$
\item $(3x - 2y) + (4x - 2y) = 3x - 2y + 4x - 2y = 7x - 4y$
\item $9y - (5y - 3) = 9y - 5y + 3 = 4y + 3$
\item $13x - (6x - 4) = 13x - 6x + 4 = 7x + 4£$ 
\end{enumerate}
\end{absolutelynopagebreak}

\begin{absolutelynopagebreak}
\noindent {\bf 1115 }
\begin{enumerate}[a)]
\item $3x = x + 16$\\
$3x - x = x - x+ 16$\\
$2x  = 16$\\
$x  = 8$
\item $7y = 15 + 2y$\\
$7y -2y = 15 + 2y - 2y$\\
$5y = 15$\\
$y = 3$
\item $x + 20 = 5x$\\
$x - x + 20 = 5x -x$\\
$x - x + 20 = 5x -x$\\
$20 = 4x$\\
$5 = x$\\
$x = 5$
\item $5y - 7 = 2y + 11$\\
$5y - 2y - 7  + 7 = 2y - 2y + 11 + 7$\\
$3y = 18$\\
$y = 6$\\ 
\end{enumerate}
\end{absolutelynopagebreak} 

\begin{absolutelynopagebreak}
\noindent {\bf 1139 }
\begin{enumerate}[a)]
\item $f(2) = 10 - 3 \cdot 2 = 10 - 6 = 4$
\item $f(0) = 10 - 3 \cdot 0 = 10 - 0 = 10$
\item $f(0) = 10 - 3 \cdot (-2) = 10 + 6 = 16$
\item $f(0) = 10 - 3 \cdot 2a = 10 - 6a$
\end{enumerate}
\end{absolutelynopagebreak} 

\begin{absolutelynopagebreak}
\noindent {\bf 1143 }
\begin{enumerate}[a)]
\item $f(x) = x^2 + 3$
\item $f(x) = (x - 3)^2$
\end{enumerate}
\end{absolutelynopagebreak}

\begin{absolutelynopagebreak}
\noindent {\bf 1203 }
\begin{enumerate}[a)]
\item $k = 5, m = 3$
\item $k = -2, m = 1$
\item $k = 1, m = 0$
\item $k = 0, m = 4$
\end{enumerate}
\end{absolutelynopagebreak}

\begin{absolutelynopagebreak}
\noindent {\bf 1209 }
\begin{enumerate}[a)]
\item Linje A har störst k-värde då denna linje lutar mest.
\item Linje D har minst k-värde då denna linje lutar minst.
\item Linje B har störst m-värde då denna linje ligger högst när x är 0.
\end{enumerate}
\end{absolutelynopagebreak}

\begin{absolutelynopagebreak}
\noindent {\bf 1218 }
\begin{enumerate}[a)]
\item $\Delta x = 5 - 1 = 4$
\item $\Delta y = 5 - 3 = 2$
\item $k = \dfrac{\Delta y}{\Delta x} = \dfrac{4}{2} = 2$
\end{enumerate}
\end{absolutelynopagebreak}

\begin{absolutelynopagebreak}
\noindent {\bf 1235 }
\begin{enumerate}[a)]
\item Två linjer är parallella om och endast om linjerna har samma riktningskoefficient k. Då den kända linjen har riktning koefficienten 2 så 
är även den sökta rikningskoefficienten 2.
\item Två linjer, A och B, är vinkxelräta om och endast om $k_{A} \cdot k_{B} = -1$ Den sökta riktningskoefficienten blir därvid $ k = \dfrac{-1}{2} = -0,5$
\end{enumerate}
\end{absolutelynopagebreak}

\begin{absolutelynopagebreak}
\noindent {\bf 1244 }
\begin{enumerate}[a)]
\item $k = 5$ och $(x_{1}, y_{1}) = (3, 4)$  vi använder detta i enpunktsformeln\\
$y - 4 = 5(x - 3)$\\
$y - 4 = 5x - 11$\\
$y = 5x - 19$
\item $k = 5$ och $(x_{1}, y_{1}) = (-2, 6)$  vi använder detta i enpunktsformeln\\
$y - 6 = 5(x - (-2))$\\
$y - 6 = 5(x + 2)$\\
$y - 6 = 5x + 10)$\\
$y = 5x + 16$

\end{enumerate}
\end{absolutelynopagebreak}

\end {document}